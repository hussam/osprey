\documentclass[a4paper,twocolumn]{article}

\title{
  It's Never Too Late To Learn\\
  \emph{Using Online Machine-Learning in Distributed Systems}
}
\author{}
\date{}

\begin{document}
\maketitle

\section{Motivation}

Building a modern distributed service that guarantees service-level objectives
is hard. Complexity is introduced by the deployment environment and exacerbated
as the service evolves with time to accommodate hardware/environment changes or
support new features.
For example, to build a simple caching layer that guarantees a certain response
latency and throughput, developers need to take into account the number of
machines available to run that service, the network connecting these machines,
the storage hierarchy within each machine, the expected workload that the
caching layer will be servicing, and interference from other applications on the
network and machines.  A change in any one of these parameters could render the
service wildly inefficient.

Traditionally, this complexity is addressed by using highly-skilled developers
to build custom systems that are optimized for specific configurations and
objective functions.  Is there a general, principled approach to dealing with
complexity in distributed systems?
In this position paper, we propose a research project to study the use of Online
Machine Learning (ML) in distributed systems.  Hand-tuning the design and
configuration of distributed systems does not scale well with increased
complexity of software design, hardware architecture, deployment environment,
and strict service-level objectives.
To highlight this, here are some examples of where we think ML can help:

\subsection*{Known unknowns}
Systems often make decisions without having complete knowledge of all the
factors that affect, or could be affected by, these decisions.  For example, a
caching system evicts items based on what it thinks future requests will look
like, even though it cannot actually know what the future will look like. Here,
decisions are made in anticipation of future events.  In other cases, this
incomplete knowledge is due to the absence of data-computation locality. For
example, a load balancer makes decisions based on what it thinks is the current
load on other machines rather than what it actually is.

In both of these situations, assumptions are made about the unknowns offline in
\emph{design-time}--resulting in different routing and caching
algorithms. A change to the \emph{run-time} environment could render these
decisions wildly inefficient. By deferring to an ML agent, these decisions can
be made online, and can be nimble in adapting to any changes to the run-time
environment.

\subsection*{Unknown unknowns}
In some situations, system designers have no way of knowing deployment details
that would impact the system's performance. For example, a system could be built
for external deployment and used by third party customers, or it could be
deployed in a multi-tenant environment where co-located services could impact
its performance.

In these cases, it is near impossible for developers to account for all the
unknown variables in design-time. However, an ML agent is able to make decisions
online based on inputs from the run-time environment.

To be clear, ML has been used in some distributed systems. However, it has not
been used in a general way to make online decisions about core distributed
systems implementation details. We advocate that developers should focus on
high-level objectives of their system, and use online ML to make to compute the
specifics (such as policies for request routing, replica placement, or failure
monitoring) at run-time based on observed metrics.



\section{Issues of ML in Systems}

Using online ML in distributed systems poses many issues and open questions.
Here are some preliminary questions that we think are worth studying:

\subsection*{Framing}
An online ML loop, like Microsoft's Multi-World Testing Decision Service, views
the external world in terms of \emph{context} (a summarization of the world's
history), \emph{actions} suggested by the agent, and \emph{rewards} collected on
these actions. How can a distributed system's current state and service-level
objectives be translated into context and rewards? In some scenarios, such as
request routing for load balancing, the translation is easy. In others, like
replica placement, the translation is hard.

\subsection*{Hierarchy}
Distributed systems are often designed in a hierarchical manner, where
subcomponents are monitoring and managing others. These components and layers of
indirection simplify reasoning about the system at large. However, it is not
clear how hierarchy will affect an ML agent.
\begin{itemize}
  \item Will hierarchy aid the learning system via parallelism? Or will it hinder
    it since not all features are visible at every level?
  \item Should learning happen in each component of the system? Or should it be
    centralized?
  \item If learning is happening at multiple levels of the hierarchy, will that
    create interference between the learners? Or will they complement each
    other?
\end{itemize}

\subsection*{Locality}
Decisions made by an ML agent in distributed system are likely to rely on input
data collected from diverse and physically distributed components. As a result,
the input to the learning loop might be stale or inconsistent. What could be
done to mitigate those issues while making reasonable real-time decisions?
Additionally, the decisions outputted by the learning agent need to be executed
by diverse and distributed components. How could routine distributed systems
issues such failures and delays in an asynchronous setting affect the learning
system?

\subsection*{Feature selection}
What features of the run-time system are important to the ML agent? How can they
be captured effectively? Can the ML agent learn from incomplete data? For
example, if an ML agent needs to make decisions regarding performance
optimization, would ignoring garbage collection activities or the VMM result in
corrupting the learned model? How sensitive should the learner be?

\subsection*{Training}
An ML agent can be trained offline with pre-labeled data. However, is this
possible in the context of distributed systems? An online learning systems is
more logical in this context, however, how do we quantify the costs of
exploration vs exploitation of learned strategies?


\end{document}


