\documentclass[a4paper,twocolumn]{article}

\usepackage[top=1in, bottom=1in, left=1in, right=1in]{geometry}

\newcommand{\ignore}[1]{}

\title{
  It's Never Too Late To Learn\\
  \emph{Using Online Machine Learning in Distributed Systems}
}
\author{}
\date{}

\begin{document}
\maketitle

\section{Motivation}

Building a modern distributed service that guarantees service-level objectives
is hard. Complexity is introduced by the deployment environment and exacerbated
as the service evolves with time to accommodate hardware/environment changes or
support new features.
For example, to build a simple caching layer that guarantees a certain hit rate,
response latency, and throughput, developers need to take into account the expected
workload hitting the cache, the number of machines available to run it, the network
connecting these machines, the storage hierarchy within each machine, and interference
from other applications running on the network and machines.  A change in any one
of these factors could render the service inefficient.

Traditionally, this complexity is addressed by using highly-skilled developers
to build custom systems that are optimized for specific configurations and
objective functions.  Is there a general, principled approach for dealing with
complexity in distributed systems? Hand-tuning the design and configuration of a system
does not scale well with increased complexity and changes in software design, hardware
architecture, deployment environment, and service-level objectives.

We propose to use online learning to cope with this complexity. Specifically, we
view a {\em decision} made by a distributed system in the framework of
contextual online learning with partial feedback. This is a natural fit: we typically
have plenty of context surrounding a decision (which we may not know how to use), the decision
is made online, and we only receive feedback for the decision that was made. Thus, we can
replace a hand-designed policy with a new policy that has been optimized by a contextual
learning system. By supplying rich contexts to this policy, we can capture the complexity of
our environment without having to understand it; by optimizing the policy continuously, we can
can adapt to changes in the environment or service over time.

Machine learning has certainly been used in the past to optimize system decisions. Most
of this use has been restricted to offline settings with full information: for example,
supervised learning is often used to train models on data that has been annotated with
correct labels. Some systems work in an online setting with full information: for example,
a cache that applies an experts learning framework to dynamically switch between existing
(hand-designed) cache eviction policies, by running all policies in parallel and measuring
their performance. Systems that work in an online setting with partial information tend to
be for news or ad recommendation, but examples from systems infrastructure exist: for example,
caching dynamic query content [], tuning parameters of multicore data structures [], allocating
resources to cluster computing tasks [], and allocating servers to web applications [].
In all of these systems, the context used to make a decision, and the
decision itself, are {\em local} to a machine. This is achieved by either devising a decentralized
algorithm in the first place, or giving up on global optimality. Thus we are not aware of any
system that makes ``global'' decisions based on truly distributed architecture or state.

And yet we use hand-designed policies to make these decisions all the time, such as for request
routing, replica placement, replica selection, cloud resource allocation, failure recovery tasks, 
and others. What is it about distributed systems that makes it difficult to apply online learning
at standard decision points? This proposal identifies fundamental challenges posed by distributed
systems that must be analyzed and addressed in order to make online learning a standard tool for
cloud optimization.  These challenges hold even if we assume that the online learning system is
a black-box that delivers optimal decisions with zero latency--in other words,
they are fundamental to the way we design distributed systems.  We begin by highlighting some of the unknown factors
that contribute to these challenges, and then list the challenges themselves.

\subsection*{Known unknowns}

Systems often make decisions without complete knowledge of the factors that affect,
or could affect, those decisions.  For example, a caching system evicts items based
on what it believes future requests will look like, without actually knowing the future.
In other cases, incomplete knowledge is due to the absence of data-computation locality.
For example, a load balancer makes decisions based on what it thinks the current load on
machines is, e.g. based on data it collected recently, rather than the actual load at that
very instant in time.

In both of these situations, assumptions are made about the unknowns offline during
\emph{design time}, resulting in different routing and caching algorithms. A change to
the \emph{run-time} environment could easily render these choices suboptimal.

\ignore{
Online learning has the potential to make these decisions online, based on current contexts,
and can be nimble in adapting to any changes to the run-time
environment.
}

\subsection*{Unknown unknowns}

In some situations, system designers have no way of knowing deployment details
that might impact a system's performance. For example, a system could be
built for external deployment and used by third-party customers, or it could be
deployed in a multi-tenant environment where co-located services could
affect its performance in unexpected ways. 

In these cases, it is nearly impossible for developers to account for all 
unknown variables at design time. 

\ignore{
However, an ML agent is able to make
decisions online based on inputs from the run-time environment.
}
\ignore{
To be clear, ML has been used in some distributed systems. However, it has not
been used in a general way to make online decisions about core distributed
systems implementation details. We advocate that developers should focus on
high-level objectives of their system, and use online ML to make to compute the
specifics (such as policies for request routing, replica placement, or failure
monitoring) at run-time based on observed metrics.
}

\section{Using Online Learning in Distributed Systems}

As mentioned earlier, we adopt the framework of contextual online learning
with partial feedback.  To separate the systems concerns we are trying to
highlight from machine learning concerns, we abstract the learning system as a
black-box and assume it provides optimal decisions with zero latency. For a
given decision over a discrete action space, the interface to the learning
system is as follows: given a {\em context} that summarizes the state of the
world, the learning system chooses an {\em action} to take, which our system
does and later reports a {\em reward} indicating how good the action was. An
example of a learning system that supports this interface is the Decision
Service~\footnote{http://aka.ms/mwt}. An example of a distributed systems
decision is routing requests at a load balancer: the context could be load
information of each machine in the cluster, the action could be one of the
possible machines, and the reward could be the latency of processin the request.

Thus we have reduced our concerns to generating an appropriate context,
executing an action (told to us by the learnin system), and reporting reward
information.  We believe that fundamental properties of distributed systems make
even these tasks difficult, as described below.

\subsection*{Locality}

Decisions in a distributed system often rely on contextual information that
resides on diverse and physically disparate components. This information may not
be readily available at the point at which the decision is made (e.g., the load
balancer).  As a result, the context supplied to the learning system may be
stale. What are the effects of staleness, and can the learning system cope with
them (e.g., by relying on ''smoothness'' properties of the context)?  Can we
leverage recent hardware or software techniques to reduce or eliminate
staleness?

Another disparity occurs between the place a decision is made (e.g., the load
balancer) and where the resulting action is taken (e.g., the server selected to
process a request). This means that the enforcement of the action cannot be
guaranteed by the decision maker, and indeed it may be altered or reversed.
Contextual learning systems are capable of tolerating downstream logic
that alters the decision, provided certain independence assumptions
aren't violated and rewards are reported accurately. Are there circumstances in
which distributed systems would violate these assumptions enough to affect
performance?

While reward information is also not available at the decision location, this is
the norm in contextual learning settings, where reward information is only
discernible after a delay and often by a separate subsystem. Thus, we do not
expect this to pose a problem. (The faster reward information is collected, the
faster the resulting datapoint can be used by the learning subsystem.)

\subsection*{Failures}

The disparity in locality above results in a lack of fate sharing between the
distributed system components involved in a decision.  Failures in any of
htese components have implications on the inputs to the learning system. For
example, server failures in a load balanced cluster could mean that certain
context information is staler than others or unavailable altogether. How do we
convey this ''lack of information'' to the learning system, and how does it
cope with sporadically incomplete data? Should failures be encoded in the
context somehow?

Failures in distributed systems are hard to distinguish from asynchrony in some
cases. What other ways does asynchrony affect the learning system?

There is another failure mode in which important contextual information is
simply not captured due to designer ommision. In general contextual learning
algorithms scale very well with the size of the size of the context, so it is
generally a good idea to include more data than necessary. In some cases,
however, it may not be easy to access this data, e.g. obtaining statistics
about a background garbage collection process or about a VMM.

\subsection*{Hierarchy}

Distributed systems are often designed in a hierarchical manner, where
subcomponents monitor and manage others. These components and layers of
indirection simplify reasoning, building, and maintaining the system at
large. However, it is not clear how hierarchy should be handled with respect to
optimizing decisions via contextual learning. In general contextual learning is
composable, in that decisions of subcomponents can be used to influence
higher-level components, downstream components can make decisions influenced by
the decisions of upstream components, and so on.  Should the context for each
decision be limited to the level it is at in the hierarchy, or is it better for
contexts to span multiple (or all) levels? Should decision points roughly mimic
the hierarchy of the system? Consider deep memory hierarchy, for example. One
could imagine that by capturing contextual information across the entire
hierarchy, one could decide at what level to cache a piece of data, instead of
only deciding on the current level and relying on evictions to move between
levels.

\ignore{
\begin{itemize}
  \item Will hierarchy aid the learning system via parallelism? Or will it hinder
    it since not all features are visible at every level?
  \item Should learning happen in each component of the system? Or should it be
    centralized?
  \item If learning is happening at multiple levels of the hierarchy, will that
    create interference between the learners? Or will they complement each
    other?
\end{itemize}
}

\subsection*{Randomization}

Contextual learning uses exploration (controlled randomization) to discover
potentially good strategies. This is the famous explore-exploit tradeoff and is
well established to be necessary for proper online optimization. Randomization
can be encapsulated and handled by the learning system, but there are two
situations in which it might emerge to the systems level. First, if multiple
components are making decisions independently, it is important to ensure that
thre randomization used by each (e.g. the seed passed to the learning system)
are independent of each other. Second, many distributed systems already use
randomization inherently, such as randomized load balancing or randomized cache
eviction strategies. Can this randomization be leveraged by the learning
system to achieve its exploration goals naturally?

\ignore{
An ML agent can be trained offline with pre-labeled data. However, is this
possible in the context of distributed systems? An online learning systems is
more logical in this context, however, how do we quantify the costs of
exploration vs exploitation of learned strategies?
}

\ignore{
\section{Expanding the scope}

There are two drawbacksTwo drawbacks: opacity and scope.

Opacity (also, which features matter?)
- We are also pursuing work on cracking models open to see what they are doing inside.
There are different ways to gain insight as to what a model is doing. For example a simple
linear model may learn a vector of weights on the supplied features. These weights can be
studied, and indeed some techniques such as lasso are oftne used to reduce the number of features. However, these
techniques need to be handled carefully. Just becasue a feature has 0 weight doesn't mean it
can be removed: perhaps it is unimportant preceisely because it exists, and its removal all of
asudden changes the weight on other features.
}

\ignore{
contextual multi-armed bandits. idea is to provide a context, get an action,
and only observe reward for that action. This matches the dist systems world well. It will generate new
policies using ML that we could not come up with by just thinking, and may not
even be able to describe. We argue in favor of opacity and rely instead on the guarantees
provided by the ML optimization.
}


Scope:
The above replaces the policy behind a decision with one backed by an ML learning system.
Indeed systems like hte DS were designed to intervene at the decision making point in a
slim way. The above outlines the issues which tackled would result in ML based policies,
new ones. However, once the above has been mastered, there is the question of whether ML
can influence distributed systems design at a higher level. For example, can it suggest
different mechanisms for achieving a goal, or perhaps a different design of the system in the first place?

\end{document}
